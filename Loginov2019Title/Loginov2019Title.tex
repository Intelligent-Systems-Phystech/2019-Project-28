\documentclass[12pt,twoside]{article}
\usepackage{jmlda}
%\NOREVIEWERNOTES
\title
    %[Образец оформления статьи для публикации] % Краткое название; не нужно, если полное название влезает в~колонтитул
    {Мультимоделирование как универсальный способ описания выборки общего вида}
\author
    %  [Логинов~Р.\,А.] % список авторов для колонтитула; не нужен, если основной список влезает в колонтитул
    {Логинов~Р.\,А., Адуенко~А.\,А., Стрижов~В.\,В.} % основной список авторов, выводимый в оглавление
    % [Логинов~Р.\,А., Адуенко~А.\,А., Стрижов~В.\,В.] % список авторов, выводимый в заголовок; не нужен, если он не отличается от основного
\thanks
    {Работа выполнена при финансовой поддержке РФФИ, проект \No\,00-00-00000.
   Научный руководитель:  Стрижов~В.\,В.
   Задачу поставил:  Стрижов~В.\,В.
    Консультант:  Адуенко~А.\,А.}
\email
    {logipamar@yandex.ru}
%\organization
%    {$^1$Организация; $^2$Организация}
\abstract
    {В случае неоднородных данных в машинном обучении использования одной модели недостаточно. Для выявления этого используют комбинации нескольких моделей - мультимодели. Работа нацелена на то, чтобы изучить по последовательности постепенно приходящих данных эволюцию представлений о модели. Рассмотреть, в какие моменты предпочтительнее разветвлять одну модель, а также какими критериями пользоваться для объединения ещё недообученного ответвления с имеющимися моделями. Исследования проводятся на синтетических данных из многоуровневой модели или смеси распределений.

\bigskip
\textbf{Ключевые слова}: \emph {мультимоделирование, бинарная классификация}.}
\titleEng
    {JMLDA paper example: file jmlda-example.tex}
\authorEng
    {Author~F.\,S.$^1$, CoAuthor~F.\,S.$^2$, Name~F.\,S.$^2$}
\organizationEng
    {$^1$Organization; $^2$Organization}
\abstractEng
    {This document is an example of paper prepared with \LaTeXe\
    typesetting system and style file \texttt{jmlda.sty}.

    \bigskip
    \textbf{Keywords}: \emph{keyword, keyword, more keywords}.}
\begin{document}
\maketitle
%\linenumbers

\section{Введение}

Работа посвящена исследованию решения задач бинарной классификации при помощи мультимоделей. Решение этой задачи может быть использовано в вопросах кредитного скоринга [], медицинской диагностики [], предсказания качества продукции и других областях. В некоторых задачах встречаются данные, для описания которых требуется вводить несколько моделей. Например, для задачи кредитного скоринга важность признаков в модели может отличаться в зависимости от региона заявителя. Как пример, многодетность может быть положительным параметром для более благополучных регионов и отрицательным для менее состоятельных. Тогда используют решающее правило о разделении выборки на кластеры, а затем на каждом из них строят отдельную модель. Такой подход называют многоуровневой моделью.

Один из алгоритмов построения и обучения оптимальной модели, основанный на байесовском выводе и ЕМ-алгоритме, описан в [Aduenko-main]. Более того, известна процедура выявления максимального числа необходимых моделей, а также построена функция, которая, в отличие от методов, основанных на дивергенциях Брегмана и KL-дивергенциях, позволяет оценить различимость двух моделей.

Однако этот алгоритм рассматривает выборку как статическую и известную заранее. В прикладных задачах появляются данные, имеющие временную структуру. Из-за этого на одном и том же объекте ответ с течением времени может различаться. Таким образом целью работы является исследование эволюции модели во времени. Более того, в статье приведены эксперименты, выявляющие необходимый размер выборки, которую возможно отделить для построения новой модели. Подобные результаты планируется получить на синтетических данных, где временная структура будет различной: случайный выбор модели и непрерывные отрезки во времени, на которых поступает каждая модель.

В упомянутой работе такая оптимизация алгоритма не приведена, и новый подход позволяет улучшить качество мультимоделей в задачах бинарной классификации.  

Предлагается построить модификацию этого алгоритма, взяв его за основу. Один из методов заключается в том, чтобы для новой модели учитывать лишь ту часть выборки, которая поступила последней. Размер этой части является гиперпараметром и будет подобран на синтетических данных.

Затем полученный алгоритм предлагается сравнить с уже имеющимся как на построенных синтетических данных, так и на собранных в репозитории UCI данных о кредитном скоринге, бинаризованных данных о стоимости квартир и о качестве вина [].

\section{Постановка задачи}

Как уже отмечено, в работе рассматривается задача бинарной классификации. Это означает что изначально имеется некоторая выборка объектов. Объект представляется в виде пары $(\mathbf{x}, y)$, где $\mathbf{x} \in \mathbb{X} \subset \mathbb{R}^n$ - признаковое описание объекта, а $y \in \{0, 1\}$ - корректный класс объекта.

Соответственно, выборка обозначается $\mathfrak{D} = \{(\mathbf{x}_i, y_i)\}, i \in \mathcal{I} = \{1, ..., m\}$, а матрицей $\mathbf{X}$ обозначим матрицу объектов, у которой в строке с индексом $i$ будет содержаться признаковое описание объекта $\mathbf{x}_i$.

\begin{Def}
	\emph{Моделью бинарной классификации} будем называть параметрическое семейство функций $f$, отображающих декартово произведение множества значений признакового описания объектов $\mathbb{X}$ и множества параметров $\mathbb{W}$ в множество значений целевой переменной $\mathbb{Y} = \{0, 1\}$
\end{Def}

\begin{Def}
	\emph{Бинарным классификатором} называется отображение $f$ из множества признакового описания объектов $\mathbb{X}$ в пространство целевой переменной $\mathbb{Y}$
\end{Def}

\begin{Def}
	\emph{Вероятностным классификатором} называется условное распределение вида
	$$ q(y|\mathbf{x})\ :\ \mathbb{Y} \times \mathbb{X} \rightarrow \mathbb{R}^{+}$$
\end{Def}

Имея вероятностный классификатор, можно построить бинарный классификатор по следующему принципу:

$$f(\mathbf{x}) = I\{q(1|\mathbf{x}) > T\},$$

где $T \in (0, 1)$ - \emph{порог классификации}, который является одним из гиперпараметров модели.

Для оценки качества модели в общем случае необходима функция ошибки, не зависящая от порога классификации. Для этого подходит AUC ROC - площадь под ROC-кривой, описание которой имеется в []

Чтобы избежать проблемы переобучения, множество объектов представляется в виде объединения обучающей и тестовой выборки: $\mathfrak{D} = \mathfrak{D}_{test} \sqcup \mathfrak{D}_{train}$.

Соответственно определяются и множества индексов:

$$ \mathfrak{D}_{train} = \{(\mathbf{x}_i, y_i)\}, i \in \mathcal{I}_{train} \subset \mathcal{I} $$
$$ \mathfrak{D}_{test} = \{(\mathbf{x}_i, y_i)\}, i \in \mathcal{I}_{test} \subset \mathcal{I} $$

Если же объекты отсортированы во времени, то обучающая выборка должна идти во времени раньше, поэтому рассматриваются следующие множества в качестве индексных:

$$ \mathcal{I}_{train} = \{1, ..., t\} $$
$$ \mathcal{I}_{test} = \{t + 1, ..., m\} $$

\section{Введение(пример)}
После аннотации, но перед первым разделом,
располагается введение, включающее в себя
описание предметной области,
обоснование актуальности задачи,
краткий обзор известных результатов,
и~т.\,п~\cite{author09anyscience,myHandbook,author09first-word-of-the-title,voron06latex,author-and-co2007,Lvovsky03}.

\section{Название раздела}
Данный документ демонстрирует оформление статьи,
подаваемой в электронную систему подачи статей \url{http://jmlda.org/papers} для публикации в журнале <<Машинной обучение и анализ данных>>.
Более подробные инструкции по~стилевому файлу \texttt{jmlda.sty}
и~использованию издательской системы \LaTeXe\
находятся в~документе \texttt{authors-guide.pdf}.
Работу над статьёй удобно начинать с~правки \TeX-файла данного документа.

\paragraph{Название параграфа.}
%Первый раздел может содержать формальную постановку задачи,
%основные определения и~обозначения,
%известные факты, необходимые для понимания основных результатов работы,
%и~т.\,п.
Нет ограничений на~количество разделов и~параграфов в~статье.
Разделы и~параграфы не~нумеруются.

\paragraph{Теоретическую часть работы} желательно структурировать
с~помощью окружений
Def, Axiom, Hypothesis, Problem, Lemma, Theorem, Corollary, State, Example, Remark.

\begin{Def}
    Математический текст \emph{хорошо структурирован},
    если в~нём выделены определения, теоремы, утверждения, примеры, и~т.\,д.,
    а~неформальные рассуждения (мотивации, интерпретации)
    вынесены в~отдельные параграфы.
\end{Def}

\begin{State}
    Мотивации и~интерпретации наиболее важны для понимания сути работы.
\end{State}

\begin{Theorem}
    Не~менее $90\%$ коллег, заинтересовавшихся Вашей статьёй,
    прочитают в~ней не~более~$10\%$ текста.
\end{Theorem}

\begin{Proof}
    Причём это будут именно те~разделы, которые не содержат формул.
\end{Proof}

\begin{Remark}
    Выше показано применение окружений
    Def, Theorem, State, Remark, Proof.
\end{Remark}

\section{Некоторые формулы}

Образец формулы: $f(x_i,\alpha^\gamma)$.

Образец выключной формулы без номера:
\[
    y(x,\alpha) =
    \begin{cases}
        -1, & \text{если } f(x,\alpha)<0;  \\
        +1, & \text{если } f(x,\alpha)\geq 0.
    \end{cases}
\]

Образец выключной формулы с номером:
\begin{equation}
\label{eq:cases}
    y(x,\alpha) =
    \begin{cases}
        -1, & \text{если } f(x,\alpha)<0;  \\
        +1, & \text{если } f(x,\alpha)\geq 0.
    \end{cases}
\end{equation}

Образец выключной формулы, разбитой на две строки с~помощью окружения align:
\begin{align}
    R'_N(F)
        = \frac1N \sum_{i=1}^N
        \Bigl(
            & P(+1\cond x_i) C\bigl(+1,F(x_i)\bigr)+{}
        \notag % подавили номер у первой строки
    \\ {}+{}
            & P(-1\cond x_i) C\bigl(-1,F(x_i)\bigr)
        \Bigr).
        \label{eq:R(F)}
\end{align}

Образцы ссылок: формулы~\eqref{eq:cases} и~\eqref{eq:R(F)}.

\section{Пример илюстрации}

Рисунки вставляются командой \verb|\includegraphics|,
желательно с~выравниванием по~ширине колонки: \verb|[width=\linewidth]|.

Практически все популярные пакеты рисуют графики с подписями, которые трудно читать на бумаге и на слайдах из-за малого размера шрифта. Шрифт на графиках (подписи осей и цифры на осях) должны быть такого же размера, что и основной текст.

При значительном количестве рисунков рекомендуется группировать иx в одном окружении \verb|{figure}|, как это сделано на рис.~\ref{fg:Example}.

\section{Пример таблицы}
Подпись делается \emph{над таблицей}, см.~таблицу~\ref{TabExample}.


\begin{table}[t]%\small
    \caption{Подпись размещается над таблицей.}
    \label{TabExample}
    \centering\medskip%\tabcolsep=2pt%\small
    \begin{tabular}{lrrr}
    \headline
        Задача
            & \multicolumn{1}{c}{CCEL}
            & \multicolumn{1}{c}{boosting} \\
    \headline
        {\tt Cancer}
            & $\mathbf{3.46}  \pm 0.37$ (3.16)
            & $4.14 \pm 1.48$ \\
        {\tt German}
            & $\mathbf{25.78} \pm 0.65$ (1.74)
            & $29.48 \pm 0.93$ \\
        {\tt Hepatitis}
            & $18.38 \pm 1.43$ (2.87)
            & $19.90 \pm 1.80$ \\
    \hline
    \end{tabular}
\end{table}

\section{Заключение}
Желательно, чтобы этот раздел был, причём он не~должен дословно повторять аннотацию.
Обычно здесь отмечают,
каких результатов удалось добиться,
какие проблемы остались открытыми.


\bibliographystyle{unsrt}
\bibliography{jmlda-bib}
\begin{thebibliography}{1}

\bibitem{author09anyscience}
    \BibAuthor{Author\;N.}
    \BibTitle{Paper title}~//
    \BibJournal{10-th Int'l. Conf. on Anyscience}, 2009.  Vol.\,11, No.\,1.  Pp.\,111--122.
\bibitem{myHandbook}
    \BibAuthor{Автор\;И.\,О.}
    Название книги.
    Город: Издательство, 2009. 314~с.
\bibitem{author09first-word-of-the-title}
    \BibAuthor{Автор\;И.\,О.}
    \BibTitle{Название статьи}~//
    \BibJournal{Название конференции или сборника},
    Город:~Изд-во, 2009.  С.\,5--6.
\bibitem{author-and-co2007}
    \BibAuthor{Автор\;И.\,О., Соавтор\;И.\,О.}
    \BibTitle{Название статьи}~//
    \BibJournal{Название журнала}. 2007. Т.\,38, \No\,5. С.\,54--62.
\bibitem{bibUsefulUrl}
    \BibUrl{www.site.ru}~---
    Название сайта.  2007.
\bibitem{voron06latex}
    \BibAuthor{Воронцов~К.\,В.}
    \LaTeXe\ в~примерах.
    2006.
    \BibUrl{http://www.ccas.ru/voron/latex.html}.
\bibitem{Lvovsky03}
    \BibAuthor{Львовский~С.\,М.} Набор и вёрстка в пакете~\LaTeX.
    3-е издание.
    Москва:~МЦHМО, 2003.  448~с.
\end{thebibliography}
% Решение Программного Комитета:
%\ACCEPTNOTE
%\AMENDNOTE
%\REJECTNOTE
\end{document}
